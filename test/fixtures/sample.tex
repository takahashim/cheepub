%% begin
\section{テスト}\hypertarget{テスト}{}\label{テスト}

これは\textbf{テスト}です。

\subsubsection{節見出し}\hypertarget{節見出し}{}\label{節見出し}

\_節\_の\href{https://example.jp/}{テスト}です。

\begin{lstlisting}[showspaces=false,showtabs=false,language=ruby,basicstyle=\ttfamily\footnotesize,columns=fixed,frame=tlbr]
class Hello
  # say hello
  def hello
    p "hello!"
  end
end

\end{lstlisting}   %  class="language-ruby"



↓Markdownとして解釈するとただの\texttt{\textless{}hr /\textgreater{}}になります。Cheepub::Contentで解釈するとページを分割します。

\begin{center}
\rule{3in}{0.4pt}
\end{center}

\section{続き}\hypertarget{続き}{}\label{続き}

\ruby[g]{続}{つづ}きの\ruby[g]{章}{しょう}です。\rensuji{12}といった\ruby[g]{縦中横}{たてちゅうよこ}も書けます。

\subsection{箇条書き}\hypertarget{箇条書き}{}\label{箇条書き}

\begin{itemize}
\item{} 1


\begin{itemize}
\item{} 1-1
\item{} 1-2
\end{itemize}
\item{} 2
\item{} 3


\begin{itemize}
\item{} 3-1


\begin{itemize}
\item{} 3-1-1
\end{itemize}
\end{itemize}
\end{itemize}

\subsection{連番つき箇条書き}\hypertarget{連番つき箇条書き}{}\label{連番つき箇条書き}

\begin{enumerate}
\item{} 1


\begin{enumerate}
\item{} 1-1
\item{} 1-2
\end{enumerate}
\item{} 2
\item{} 3


\begin{enumerate}
\item{} 3-1


\begin{enumerate}
\item{} 3-1-1
\end{enumerate}
\end{enumerate}
\end{enumerate}

%% end
